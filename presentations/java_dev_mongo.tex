
\documentclass{beamer}

% Copyright 2010 by Brendan W. McAdams <bwmcadams@gmail.com>
%
% You may redistribute the content of this presentation for your own needs 
% provided you give credit to its author. In other words: ``Don't be a dick.''
% % Latex code based upon the ``Generic Presentation 15-45 minutes'' template from Beamer.


\mode<presentation>
{
  %\usetheme{nouvelle}
  %\usetheme{Hannover}
  \usetheme{Warsaw}
  %\usetheme[pageofpages=of, % word b/w current and total page count
  %          alternativetitlepage=true, % Fancy title page
  %          titlepagelogo=mongodb-logo]{Torino}
  %\usecolortheme{freewilly}
  %          
  %%          
  \setbeamercovered{transparent}
}


\usepackage[english]{babel}
% or whatever

\usepackage[latin1]{inputenc}
% or whatever

\usepackage{times}
\usepackage[T1]{fontenc}
% Or whatever. Note that the encoding and the font should match. If T1
% does not look nice, try deleting the line with the fontenc.


\title[Java Development with MongoDB] % (optional, use only with long paper titles)
{Java Development with MongoDB}

%\subtitle
%{Presentation Subtitle} % (optional)

\author[Brendan W. McAdams] 
{Brendan W. McAdams\\
%{\scriptsize @rit ({\bf twitter}) }
}
% - Use the \inst{?} command only if the authors have different
%   affiliation.

\institute[Novus Partners, Inc.] % (optional, but mostly needed)
{ Novus Partners, Inc.\\
  http://novus.com }
% - Use the \inst command only if there are several affiliations.
% - Keep it simple, no one is interested in your street address.

\date[Mongo NYC, May 2010] % (optional)
{May 2010 / Mongo NYC}

\subject{MongoDB Java Development}
% This is only inserted into the PDF information catalog. Can be left
% out. 



% If you have a file called "university-logo-filename.xxx", where xxx
% is a graphic format that can be processed by latex or pdflatex,
% resp., then you can add a logo as follows:
\logo{\includegraphics[width=1.5cm]{novus-logo}}



% Delete this, if you do not want the table of contents to pop up at
% the beginning of each subsection:
\AtBeginSubsection[]
{
  \begin{frame}<beamer>{Outline}
    \tableofcontents[currentsection,currentsubsection]
  \end{frame}
}


% If you wish to uncover everything in a step-wise fashion, uncomment
% the following command: 

\beamerdefaultoverlayspecification{<+->}


\begin{document}

\begin{frame}
  \titlepage
\end{frame}

\begin{frame}{Outline}
  \tableofcontents
  % You might wish to add the option [pausesections]
\end{frame}


% Since this a solution template for a generic talk, very little can
% be said about how it should be structured. However, the talk length
% of between 15min and 45min and the theme suggest that you stick to
% the following rules:  

% - Exactly two or three sections (other than the summary).
% - At *most* three subsections per section.
% - Talk about 30s to 2min per frame. So there should be between about
%   15 and 30 frames, all told.

\section{MongoDB + Java Basics}

\subsection[Getting Started]{Setting up your Java environment}

\begin{frame}{Adding the MongoDB Driver To Your Project}
 
  Assuming you're using a dependency manager, make setup simple\dots
  \begin{itemize}
    \pause
  \item
       Maven Dependency\\
      \texttt{\scriptsize
       <dependency>\\
        <groupId>org.mongodb</groupId>\\
        <artifactId>mongo-java-driver</artifactId>\\
        <version>1.4</version>\\
      </dependency>}
  \item
      Ivy Dependency\\
      \texttt{\scriptsize <dependency org="org.mongodb" name="mongo-java-driver" rev="1.4"/>}
  \end{itemize}
\end{frame}

\subsection[Connecting to MongoDB]{Connecting to MongoDB}

\begin{frame}{Simple Connection}
    Getting connected to MongoDB is simple; 
    Connections are pooled, so you only need one\dots\\
    \texttt{\scriptsize import com.mongodb.Mongo;\\ import com.mongodb.DB; }
    \pause
    \begin{itemize}
        \item{\texttt{\scriptsize Mongo m = new Mongo();}}
        \item{\texttt{\scriptsize Mongo m = new Mongo(``localhost'');}}
        \item{\texttt{\scriptsize Mongo m = new Mongo(``localhost'', 27017);}}
        \item{Fetch a Database Handle (lazy)\dots\\
            \texttt{\scriptsize DB db = m.getDB(``javaDemo'');}} 
        \item{If you need to authenticate\dots\\
        \texttt{\scriptsize boolean auth = db.authenticate(``login'', ``password'');}
        }
    \end{itemize}

\end{frame}

\section{Working with MongoDB}

\subsection[Collections + Documents]{Collections + Documents}

\begin{frame}{Working with Collections}
    Collections are MongoDB ``tables''\dots
    \pause
    \begin{itemize}
        \item{List all of the collections in a database\dots \\
            \texttt{\scriptsize 
            \begin{tabbing}
            Set<String> \= colls = db.getCollectionNames();\\
            for (String s : colls) \{\\
            \> System.out.println(s);\\
            \}\\
            \end{tabbing}
            }
        }
        \item{Get a specific collection (lazy)\dots \\
            \texttt{\scriptsize 
            \begin{tabbing}
            Set<String> \= colls = db.getCollectionNames();\\
            for (String s : colls) \{\\
            \> System.out.println(s);\\
            \}\\
            \end{tabbing}
            }
        }
        \item{Count the number of documents in a collection\dots \\
            \texttt{\scriptsize coll.getCount(); }
        }

    \end{itemize}
\end{frame}

\begin{frame}{MongoDB Documents}{BSON}
    \begin{itemize}
        \item ``Documents'' are MongoDB's ``rows''. 
        \item MongoDB's Internal Document representation is '{\bf BSON}'
        \begin{itemize}
            \item {\bf BSON} is a binary optimized flavor of {\bf JSON}
            \item Corrects {\bf JSON}'s inefficiency in string encoding (Base64)
            \item Supports extras including Regular Expressions, Byte Arrays, DateTimes \& Timestamps, as well as datatypes for Javascript code blocks \& functions.
            \item Creative Commons licensed.
            \item {\bf BSON} implementation being split into its own package in most drivers.
        \end{itemize}
    \end{itemize}
\end{frame}

\begin{frame}{MongoDB Documents}{From Java: BasicDBObject}
    \begin{itemize}
        \item Java representation of {\bf BSON} is the map-like {\bf DBObject} \pause (Java 2.0 driver has a new base class of {\bf BSONObject} related to the {\bf BSON} split-off)
        \item Easiest way to work with Mongo Documents is BasicDBObject\dots
            \begin{itemize}
                \item {\bf BasicDBObject} implements {\bf java.util.LinkedHashMap<String, Object>}
                \item Mutable object
                \item Can take a {\bf Map} as a constructor parameter
                \item Example: 
                    \texttt{\scriptsize DBObject doc = new BasicDBObject();\\
                    doc.put(``username'', ``bwmcadams'');\\
                    doc.put(``password'', ``MongoNYC'');}
                \item {\em toString} returns a {\bf JSON} serialization.
            \end{itemize}
        \item Use {\bf BasicDBList} (implements {\bf java.util.ArrayList<Object>}) to represent Arrays.
    \end{itemize}
\end{frame}

\begin{frame}{MongoDB Documents}{From Java: BasicDBObjectbuilder}
    For those who prefer immutability\dots
    \begin{itemize}
        \item {\bf BasicDBObjectBuilder} follows the {\em Builder} pattern.
        \item {\em add()} your keys \& values:\\
        \texttt{
            \tiny
            BasicDBObjectBuilder builder = BasicDBObjectBuilder.start();\\
            builder.add(``username'', ``bwmcadams'');\\
            builder.add(``password'', ``MongoNYC'');\\
            builder.add(``presentation'', ``Java Development with MongoDB'');
        }
        \item A {\bf BasicDBObjectBuilder} is not a {\bf DBObject}. 
        \item Call {\em get()} to return the built-up {\bf DBObject}.
        \item {\em add()} returns itself so you can chain calls instead: \texttt{\tiny BasicDBOBjectBuilder.start().add(``username'', ``bwmcadams'').add(``password'', ``MongoNYC'').add(``presentation'', ``Java Development with MongoDB'').get();}
    \end{itemize}
\end{frame}

\begin{frame}{MongoDB Documents}{Implementation by Extension: DBObject}
    \begin{itemize}
        \item If you want to create your own concrete objects, extend \& implement {\bf DBObject}
            \begin{itemize}
                \item Requires you implement a map-like interface including ability to get \& set fields by key (even if you don't use them, required to deserialize)
                \item Instances of {\bf DBObject} can be saved directly to MongoDB.
            \end{itemize}
        \item Feeling Fancy? Reflect instead\dots
            \begin{itemize}
                \item Use {\bf ReflectionDBObject} as a base class for your {\em Beans}.
                \item {\bf ReflectionDBObject} uses reflection to proxy your getters \& setters and behave like a DBObject.
                \item Downside: With Java's single inheritance you are stuck using this as your base class.
            \end{itemize}
        \item Existing Object Model? ORM-Like Solutions (\textsl{Detailed later})\dots
            \begin{itemize}
                \item Daybreak 
                \item Morphia
            \end{itemize}
              
    \end{itemize}
\end{frame}

\subsection[Inserting Documents to MongoDB]{Inserting Documents to MongoDB}

\begin{frame}{Inserting Documents}
    \begin{itemize}
        \item One at a time:\\
            \texttt{\tiny DBObject doc = BasicDBOBjectBuilder.start().add(``username'', ``bwmcadams'').add(``password'', ``MongoNYC'').add(``presentation'', ``Java Development with MongoDB'').get();\\
            // DBCollection coll\\
            coll.insert(doc);\\
            }
        \item Got multiple documents? Call {\em insert()} in a loop, or pass {\bf DBObject[]} or {\bf List<DBObject>}
        \item Three ways to store your documents:
        \begin{itemize}
            \item {\sc Insert} ({\em insert()}) always attempts to add a new row.
            \item {\sc Save} ({\em save()}) only attempts to insert unless {\em \_id} is defined.  Otherwise, it will attempt to update the identified document.
            \item {\sc Update} ({\em update()}) Allows you to pass a query to filter by and the fields to change.  Boolean option ``multi'' specifies if multiple documents should be updated.  Boolean ``upsert'' specifies that the object should be inserted if it doesn't exist (e.g. query doesn't match).
        \end{itemize}
    \end{itemize}
\end{frame}

\subsection[Querying MongoDB]{Querying MongoDB}

\begin{frame}{MongoDB Querying}{Implementation by Extension: DBObject}
    \begin{itemize}
        \item Find a single row with {\em findOne()}.  Takes the first row returned.
        \item{Getting a cursor of all documents ({\em find()} with no query):\\ 
            \texttt{\scriptsize 
                \begin{tabbing}
                DBCursor \= cur = coll.find();\\
                while (cur.hasNext()) \{\\
                \> System.out.println(cur.next());\\
                \}\\
            \end{tabbing}
            }}
    \end{itemize}
\end{frame}

\end{document}


